\documentclass[]{article}
\usepackage{tikz}
\usetikzlibrary{shapes,arrows}
\tikzstyle{line} = [draw, -latex']
\usepackage{lmodern}
\usepackage{amssymb,amsmath}
\usepackage{ifxetex,ifluatex}
\usepackage{fixltx2e} % provides \textsubscript
\ifnum 0\ifxetex 1\fi\ifluatex 1\fi=0 % if pdftex
  \usepackage[T1]{fontenc}
  \usepackage[utf8]{inputenc}
\else % if luatex or xelatex
  \ifxetex
    \usepackage{mathspec}
  \else
    \usepackage{fontspec}
  \fi
  \defaultfontfeatures{Ligatures=TeX,Scale=MatchLowercase}
\fi
% use upquote if available, for straight quotes in verbatim environments
\IfFileExists{upquote.sty}{\usepackage{upquote}}{}
% use microtype if available
\IfFileExists{microtype.sty}{%
\usepackage[]{microtype}
\UseMicrotypeSet[protrusion]{basicmath} % disable protrusion for tt fonts
}{}
\PassOptionsToPackage{hyphens}{url} % url is loaded by hyperref
\usepackage[unicode=true]{hyperref}
\hypersetup{
            pdftitle={Generacja grafu i odnalezienie najkrótszej ścieżki pomiędzy węzłami: graph},
            pdfauthor={Ulyana Petrashevich, Inga Maziarz}
            pdfborder={0 0 0},
            breaklinks=true}
\urlstyle{same}  % don't use monospace font for urls
\IfFileExists{parskip.sty}{%
\usepackage{parskip}
}{% else
\setlength{\parindent}{0pt}
\setlength{\parskip}{6pt plus 2pt minus 1pt}
}
\setlength{\emergencystretch}{3em}  % prevent overfull lines
\providecommand{\tightlist}{%
  \setlength{\itemsep}{0pt}\setlength{\parskip}{0pt}}
\setcounter{secnumdepth}{0}
% Redefines (sub)paragraphs to behave more like sections
\ifx\paragraph\undefined\else
\let\oldparagraph\paragraph
\renewcommand{\paragraph}[1]{\oldparagraph{#1}\mbox{}}
\fi
\ifx\subparagraph\undefined\else
\let\oldsubparagraph\subparagraph
\renewcommand{\subparagraph}[1]{\oldsubparagraph{#1}\mbox{}}
\fi

% set default figure placement to htbp
\makeatletter
\def\fps@figure{htbp}
\makeatother


\title{\texttt{Sprawozdanie końcowe}\\Generacja grafu i odnalezienie najkrótszej ścieżki pomiędzy węzłami - \texttt{graph}}
\author{Ulyana Petrashevich, Inga Maziarz}
\date{02.04.2022}

\begin{document}
\maketitle

\section{Opis teoretyczny zagadnienia}\label{header-n231}


\section{Opis wywołania}\label{header-n233}
Program \texttt{graph} akceptuje następujące argumenty wywołania:

\begin{itemize}
\item
  \texttt{-{}-file\ filename} nazwa pliku z danymi;
\item
  \texttt{-{}-from\ k} podawany jest numer węzła początkowego w ścieżce; domyślnie \texttt{k\ =\ 0};
\item
  \texttt{-{}-to\ l} podawany jest numer węzła końcowego w ścieżce; domyślnie \texttt{l\ =\ maksymalny numer w grafie};
\item
  \texttt{-{}-output\ filename} nazwa pliku, w który będzie zapisywana ścieżka; domyślnie \texttt{stdout};
\item
  \texttt{-{}-grow\ n} określa liczbę wierszy generowanego grafu; domyślnie \texttt{n\ =\ 10};
\item
  \texttt{-{}-gcol\ m} określa liczbę kolumn w generowanym grafie; domyślnie \texttt{m\ =\ 10};
\item
  \texttt{-{}-gfrom\ x} przyjmuje granicę dolną zakresu wartości losowanych wag, domyślnie \texttt{x\ =\ 0.01};
\item
  \texttt{-{}-gto\ y} przyjmuje granicę górną zakresu wartości losowanych wag, domyślnie \texttt{y\ =\ 10.0};
\item
  \texttt{-{}-gconnect\ s} przyjmuje flagę na spójność generowanego grafa. 0 - graf niespójny, 1 - graf spójny, 2 - graf losowy;
\item
  \texttt{-{}-goutput\ filename} nazwa pliku, w którym będą zapisane parametry wygenerowanego grafu, domyślnie \texttt{stdout}.
\end{itemize}
Dla uruchomienia algorytmu znajdowania ścieżki wystarczy podać chociażby jeden parametr z nią związany, t.j. nazwa pliku z danymi, numer węzła początkowego, numer węzła końcowego i plik wyjściowy do ścieżki. Przy niepodaniu parametru program ustali wartość domyślną lub wygeneruje graf w przypadku niepodania nazwy pliku z danymi.

Sześć końcowych parametrów odpowiadają za właściwości generowanego grafu, jak osobno od modułu odnalezienia ścieżki, tak i wewnątrz go.

Przy niepodaniu żadnych argumentów program wygeneruje graf, oparty na wartościach domyślnych.

Obowiązkowe właściwości podanego grafu:

Program będzie sprawdzał, czy graf odpowiada następnym właściwościom:
\begin{itemize}
    \item \texttt{Graf musi mieć postać „grafu-kratki”. }
    
To znaczy, że po przedstawieniu grafa w postaci graficznej, jego wierzchołki mogą być złączone krawędzią tylko z wierzchołkami sąsiednimi w osi poziomej i pionowej. Przykład:

0 – 1

|\, \, \, \,|

2 – 3  

Między wierzchołkiem 0 a 3 krawędzi być nie może.
    \item \texttt{Graf musi być spójny.}

To znaczy, że od każdego wierzchołka grafu istnieje ścieżka do każdego innego wierzchołka.

\end{itemize}

\subsection{Widok pliku}
Program jest uczulony na podanie danych w nieprawidłowym formacie. Może to skutkować niepoprawne traktowanie danych, aż do napotkania błędów, skutkujących przerywanie działania programu. 

Schemat poprawnego formatu danych:

Pierwszą linijkę zajmują dwie liczby: liczba wierszy i liczba kolumn, oddzielone od siebie minimum jedną spacją.

Zaczynając od drugiej linijki, podawane są krawędzie i ich wagi. Numer wierzchołka, od którego zaczyna się krawędź, jest równy (numeru linijki - 2), np. w 2. Linijce, tuż po rozmiarach grafu, będą opisane krawędzi, prowadzące do wierzchołka 0. Pierwsze dane o krawędzi muszą być podane, odstąpiwszy od początku linijki minimum jedną tabulację lub spację. Między numerem drugiego wierzchołka, złączonego krawędzią, a jej wagą musi być napisany chociażby jeden znak nie pusty, np :. Separatorem dziesiętnym wagi musi być kropka(.). Między dwoma krawędziami muszą być dwie lub więcej spacji. Tuż po ostatniej krawędzi, odnoszącej się do danego wierzchołka, przechodzimy na nową linijkę.

Nie ma potrzeby podawać wagi dwa razy symetrycznie, np. od 0 do 1 i od 1 do 0 – program duplikuje wagi przy wczytaniu.

Jeżeli do wierzchołka nie chcemy przypisywać żadnych krawędzi, pozostawiamy odpowiednią linijkę pustą – bez tabulacji, spacji lub innych znaków.

Przykład poprawnej formy danych:
\begin{verbatim}
2 2
    2 :2.76543  1 :4.56134

    1 :4.567345
    2 :6.47586
\end{verbatim}
W powyższym przykładzie wierzchołek numer 1 nie ma przypisanych krawędzi (choć będzie ich miał po duplikowaniu krawędzi od wierzchołków 0 i 2).
\section{Testy}\label{header-n233}

W pliku Makefile znajdują się wywołania testów programów pod nazwami „test1”, „test2” i td. Testy, związane tylko z generatorem grafu, nazywają się „testg1” i td. Dodatkowo dodany test „help”, pokazujący zachowanie programu przy podaniu nieprawidłowych opcji.

Testy, które nie da się zaimplementować do Makefile, to testy, związane z poprawnością danych w pliku. Takie testy przeprowadziłyśmy ręcznie.

\begin{enumerate}
    \item Dane wejściowe: \texttt{./graph -{}-file dane.txt -{}-from 0 -{}-to 3 -{}-output sciezka.txt}
    
Plik \texttt{dane.txt} zawiera:
\begin{verbatim}
2 2
    1 :-5.74  2 :3.21
    3 :10.21
    3 :1.74
\end{verbatim}
Na wyjściu otrzymujemy komunikat \texttt{„ Waga krawędzi między węzłem 0 i 1 jest ujemna, równa -5.740000. Ustawiam wartość 5.740000.”}, przekazany na \texttt{stdout} i plik \texttt{sciezka.txt}, zawierający:
\begin{verbatim}
Najkrótsza ścieżka: 
0 -3.210000- 2 -1.740000- 3
Długość ścieżki równa 4.950000
\end{verbatim}
    \item Dane wejściowe: \texttt{./graph -{}-file dane.txt -{}-output sciezka.txt}

Plik \texttt{dane.txt} zawiera:
\begin{verbatim}
Nie graf!
\end{verbatim}
Na wyjściu otrzymujemy komunikat \texttt{”Błędny format pliku z grafem. Przerywam działanie.”}, przekazany na \texttt{stderr}.
\end{enumerate}
\end{document}