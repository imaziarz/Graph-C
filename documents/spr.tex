\documentclass[]{article}
\usepackage{tikz}
\usetikzlibrary{shapes,arrows}
\tikzstyle{line} = [draw]
\usepackage{lmodern}
\usepackage{amssymb,amsmath}
\usepackage{ifxetex,ifluatex}
\usepackage{fixltx2e} % provides \textsubscript
\ifnum 0\ifxetex 1\fi\ifluatex 1\fi=0 % if pdftex
  \usepackage[T1]{fontenc}
  \usepackage[utf8]{inputenc}
\else % if luatex or xelatex
  \ifxetex
    \usepackage{mathspec}
  \else
    \usepackage{fontspec}
  \fi
  \defaultfontfeatures{Ligatures=TeX,Scale=MatchLowercase}
\fi
% use upquote if available, for straight quotes in verbatim environments
\IfFileExists{upquote.sty}{\usepackage{upquote}}{}
% use microtype if available
\IfFileExists{microtype.sty}{%
\usepackage[]{microtype}
\UseMicrotypeSet[protrusion]{basicmath} % disable protrusion for tt fonts
}{}
\PassOptionsToPackage{hyphens}{url} % url is loaded by hyperref
\usepackage[unicode=true]{hyperref}
\hypersetup{
            pdftitle={Generacja grafu i odnalezienie najkrótszej ścieżki pomiędzy węzłami: graph},
            pdfauthor={Ulyana Petrashevich, Inga Maziarz}
            pdfborder={0 0 0},
            breaklinks=true}
\urlstyle{same}  % don't use monospace font for urls
\IfFileExists{parskip.sty}{%
\usepackage{parskip}
}{% else
\setlength{\parindent}{0pt}
\setlength{\parskip}{6pt plus 2pt minus 1pt}
}
\setlength{\emergencystretch}{3em}  % prevent overfull lines
\providecommand{\tightlist}{%
  \setlength{\itemsep}{0pt}\setlength{\parskip}{0pt}}
\setcounter{secnumdepth}{0}
% Redefines (sub)paragraphs to behave more like sections
\ifx\paragraph\undefined\else
\let\oldparagraph\paragraph
\renewcommand{\paragraph}[1]{\oldparagraph{#1}\mbox{}}
\fi
\ifx\subparagraph\undefined\else
\let\oldsubparagraph\subparagraph
\renewcommand{\subparagraph}[1]{\oldsubparagraph{#1}\mbox{}}
\fi

% set default figure placement to htbp
\makeatletter
\def\fps@figure{htbp}
\makeatother


\title{\texttt{Sprawozdanie końcowe}\\Generacja grafu i odnalezienie najkrótszej ścieżki pomiędzy węzłami - \texttt{graph}}
\author{Ulyana Petrashevich, Inga Maziarz}
\date{02.04.2022}

\begin{document}
\maketitle

\section{Opis teoretyczny zagadnienia}\label{header-n231}
Program \texttt{graph} służy do wyszukiwania najkrótszej ścieżki w grafie ważonym nieskierowanym. Program zawiera funkcje wczytującą i wypisującą graf z/do pliku oraz generator grafu spójnego i niespójnego. Za wyszukiwanie najkrótszej ścieżki odpowiedzialne są funkcje \texttt{bfs} oraz \texttt{dijkstra}. \texttt{BFS}, czyli algorytm przeszukiwania wszerz, działa następująco: 
\begin{itemize}
\item
  Węzeł początkowy oznaczany jest jako odwiedzony. Do kolejki dodawane są sąsiadujące z nim węzły (w kolejności od węzła z najmniejszym indeksem).
\item
  Odwiedzony zostaje następny węzeł w kolejce. Postępowanie jest analogiczne do wcześniejszego; do kolejki zostają dodane węzły sąsiadujące z obecnym, jednak tylko te, które nie zostały odwiedzone wcześniej.
  
\item
  Proces jest powtarzany aż do momentu odwiedzenia wszystkich węzłów z kolejki. Jeżeli na końcu wszystkie węzły w grafie zostały odwiedzone, oznacza to, że przeszukiwany graf jest spójny. 
\end{itemize}
Jeżeli \texttt{bfs} zwróci wartość oznaczającą spójność grafu, to program przechodzi do kolejnego kroku, którym jest znalezienie najkrótszej ścieżki algorytmem Dijkstry, który działa w poniższy sposób:
\begin{itemize}
\item
Długości przy węźle początkowym otrzymują wartość 0. Długość ścieżki do każdego innego węzła zostaje ustawiona na nieskończoność. 
\item
  Oznaczamy węzeł początkowy jako odwiedzony. Dla każdego jego sąsiada zostaje przypisana długość równa wadze krawędzi między nimi. 
\item
  Nieodwiedzony węzeł o najmniejszej przypisanej długości zostaje oznaczony jako odwiedzony. Dla każdego jego sąsiada zostaje obliczona wartość równa sumie długości przy obecnym węźle i wagi krawędzi między nimi. Jeżeli znaleziona wartość jest mniejsza niż przypisana do sąsiada, to zostaje ona zamieniona. 
\item
  Poprzedni krok jest powtarzany aż do odwiedzenia wszystkich węzłów. Ostatecznie każdy węzeł (w tym wybrany jako końcowy) ma przypisaną długość najkrótszej ścieżki od węzła początkowego. Zapamiętana ścieżka może zostać wypisana na ekran.
\end{itemize}

\section{Opis wywołania}\label{header-n233}
Program \texttt{graph} akceptuje następujące argumenty wywołania:

\begin{itemize}
\item
  \texttt{-{}-file\ filename} nazwa pliku z danymi;
\item
  \texttt{-{}-from\ k} podawany jest numer węzła początkowego w ścieżce; domyślnie \texttt{k\ =\ 0};
\item
  \texttt{-{}-to\ l} podawany jest numer węzła końcowego w ścieżce; domyślnie \texttt{l\ =\ maksymalny numer w grafie};
\item
  \texttt{-{}-output\ filename} nazwa pliku, w który będzie zapisywana ścieżka; domyślnie \texttt{stdout};
\item
  \texttt{-{}-grow\ n} określa liczbę wierszy generowanego grafu; domyślnie \texttt{n\ =\ 10};
\item
  \texttt{-{}-gcol\ m} określa liczbę kolumn w generowanym grafie; domyślnie \texttt{m\ =\ 10};
\item
  \texttt{-{}-gfrom\ x} przyjmuje granicę dolną zakresu wartości losowanych wag, domyślnie \texttt{x\ =\ 0.01};
\item
  \texttt{-{}-gto\ y} przyjmuje granicę górną zakresu wartości losowanych wag, domyślnie \texttt{y\ =\ 10.0};
\item
  \texttt{-{}-gconnect\ s} przyjmuje flagę na spójność generowanego grafa. 0 - graf niespójny, 1 - graf spójny, 2 - graf losowy;
\item
  \texttt{-{}-goutput\ filename} nazwa pliku, w którym będą zapisane parametry wygenerowanego grafu, domyślnie \texttt{stdout}.
\end{itemize}
Dla uruchomienia algorytmu znajdowania ścieżki wystarczy podać chociażby jeden parametr z nią związany, t.j. nazwa pliku z danymi, numer węzła początkowego, numer węzła końcowego i plik wyjściowy do ścieżki. Przy niepodaniu parametru program ustali wartość domyślną lub wygeneruje graf w przypadku niepodania nazwy pliku z danymi.

Sześć końcowych parametrów odpowiadają za właściwości generowanego grafu, jak osobno od modułu odnalezienia ścieżki, tak i wewnątrz go.

Przy niepodaniu żadnych argumentów program wygeneruje graf, oparty na wartościach domyślnych.

Obowiązkowe właściwości podanego grafu:

Program będzie sprawdzał, czy graf odpowiada następnym właściwościom:
\begin{itemize}
    \item \texttt{Graf musi mieć postać „grafu-kratki”. }
    
To znaczy, że po przedstawieniu grafu w postaci graficznej, jego wierzchołki mogą być złączone krawędzią tylko z wierzchołkami sąsiednimi w osi poziomej i pionowej. Przykład:

0 – 1

|\, \, \, \,|

2 – 3  

Między wierzchołkiem 0 a 3 krawędzi być nie może.
    \item \texttt{Graf musi być spójny.}

To znaczy, że od każdego wierzchołka grafu istnieje ścieżka do każdego innego wierzchołka.

\end{itemize}

\subsection{Widok pliku}
Program jest uczulony na podanie danych w nieprawidłowym formacie. Może to skutkować niepoprawnym traktowaniem danych, aż do napotkania błędów, skutkujących przerwaniem działania programu. 

Schemat poprawnego formatu danych:

Pierwszą linijkę zajmują dwie liczby: liczba wierszy i liczba kolumn, oddzielone od siebie minimum jedną spacją.

Zaczynając od drugiej linijki, podawane są krawędzie i ich wagi. Numer wierzchołka, od którego zaczyna się krawędź jest równy (numerowi linijki - 2), np. w 2 linijce, tuż po rozmiarach grafu, będą opisane krawędzie prowadzące do wierzchołka 0. Pierwsze dane o krawędzi muszą być podane, odstąpiwszy od początku linijki minimum jedną tabulację lub spację. Między numerem drugiego wierzchołka, złączonego krawędzią, a jej wagą musi być napisany chociażby jeden znak niepusty, np (:). Separatorem dziesiętnym wagi musi być kropka(.). Między dwoma krawędziami muszą być dwie lub więcej spacji. Tuż po ostatniej krawędzi, odnoszącej się do danego wierzchołka, przechodzimy na nową linijkę.

Nie ma potrzeby podawać wagi dwa razy symetrycznie, np. od 0 do 1 i od 1 do 0 – program duplikuje jedną z podanych wag przy wczytaniu.

Jeżeli do wierzchołka nie chcemy przypisywać żadnych krawędzi, pozostawiamy odpowiednią linijkę pustą – bez tabulacji, spacji lub innych znaków.

Przykład poprawnej formy danych:
\begin{verbatim}
2 2
    2 :2.76543  1 :4.56134

    1 :4.567345
    2 :6.47586
\end{verbatim}
W powyższym przykładzie wierzchołek numer 1 nie ma przypisanych krawędzi (choć będzie je miał po zduplikowaniu krawędzi od wierzchołków 0 i 2).
Postać symboliczna:

2\texttt{\verbvisiblespace}2\textbackslash n

\textbackslash t2\texttt{\verbvisiblespace}:2.76543\texttt{\verbvisiblespace}\texttt{\verbvisiblespace}1\texttt{\verbvisiblespace}:4.56134\textbackslash n

\textbackslash n

\textbackslash t1\texttt{\verbvisiblespace}:4.567345\textbackslash n

\textbackslash t2\texttt{\verbvisiblespace}:6.47586EOF

\section{Testy}\label{header-n233}

W pliku Makefile znajdują się wywołania testów programów pod nazwami „test1”, „test2” itd. Testy, związane tylko z generatorem grafu, nazywają się „testg1” itd. Dodatkowo dodany test „help”, pokazujący zachowanie programu przy podaniu nieprawidłowych opcji.
\begin{itemize}
    \item help
    ./graph -h
    
    Test służy do sprawdzania działania pliku przy podaniu nieopisanych opcji. Program wyświetla komunikat i instrukcję użycia.
\begin{verbatim}
./graph: invalid option -- 'h'
Usage: ./graph [[--file data_file] [--from k] [--to l] [--output output_file]] 
    [[--grow n] [--gcol m] [--gfrom x] [--gto y] [--gconnect s]
        [--goutput data_output_file]]
    if any od first four parameters was given then
        if data_file is given then
            reads graph points from data_file
            finds path
                -from k (>= 0 and <= maximum node nr.) node (0 by default)
                -to l (>= 0 and <= maximum node nr.) node (maximum node nr.by default)
            writes the path to output_file (stdout by default)
        else (data_file not given)
            generate connected graph
                -of n (> 0) rows (10 by default)
                -of m (> 0) columns (10 by default)
                -with edge weights within <x, y> range (<0.1, 10> by default) 
            write generated graph to data_output_file (stdout by default)
            finds path
                -from k (>= 0 and <= maximum node nr.) node (0 by default)
                -to l (>= 0 and <= maximum node nr.) node (maximum node nr.by default)
            writes the path to output_file (stdout by default)
        endif
    else (none of first four parameters was given) then
        generate graph
            -of n (> 0) rows (10 by default)
            -of m (> 0) columns (10 by default)
            -with edge weights within <x, y> range (<0.1, 10> by default) 
            -connectivity defined by s: 0 - not connected, 1 - connected,
                2 - random one (2 by default)
        write generated graph to data_output_file (stdout by default)
    endif
make: *** [Makefile:8: help] Error 1
\end{verbatim}
    \item test1:
    ./graph -{}-file dane.txt -{}-from 0 -{}-to 3 -{}-output sciezka.txt
    
    Program pobiera dane z pliku przygotowanego dane.txt i szuka ścieżki od wierzchołka 0 do wierzchołka 3, zapisując ją do pliku sciezka.txt.
    
    Plik dane.txt zawiera:
\begin{verbatim}
2 3
	1 :1.0  3 :4.0
	0 :1.0  2 :3.0  4 :6.0
	1 :3.0  5 :7.0
	0 :4.0  4 :10.0
	1 :6.0  3 :10.0  5 :2.0
	2 :7.0  4 :2.0
\end{verbatim}
Po uruchomieniu testu do pliku sciezka.txt zostały dodane dane:
\begin{verbatim}
Najkrótsza ścieżka: 
0 -4.000000- 3
Długość ścieżki równa 4.000000
\end{verbatim}
\item testg1: ./graph -{}-grow 2 -{}-gcol 3 -{}-gfrom 4 -{}-gto 12 -{}-gconnect 1 -{}-goutput gen.txt

Program generuje graf o 2 wierszach, 3 kolumnach z wagami w zakresie [4,12], spójny, do pliku gen.txt.

Do pliku gen.txt zostały zapisane dane:
\begin{verbatim}
2 3
	  1 :5.193650  3 :5.579993
	  0 :5.193650  2 :7.949110  4 :5.549342
	  1 :7.949110  5 :7.208232
	  0 :5.579993  4 :8.846293
	  1 :5.549342  3 :8.846293  5 :11.670649
	  2 :7.208232  4 :11.670649
\end{verbatim}
\item testg2: ./graph -{}-grow 3 -{}-gcol 3 -{}-gfrom 1 -{}-gto 10 -{}-gconnect 0 -{}-goutput gen.txt

Program generuje graf o 3 wierszach, 3 kolumnach, z wagami w zakresie [1,10], niespójny, do pliku gen.txt

Do pliku gen.txt zostały zapisane dane:
\begin{verbatim}
3 3

	  2 :4.019620
	  1 :4.019620
	  4 :5.493351  6 :4.151664
	  3 :5.493351  5 :4.330304  7 :9.003724
	  4 :4.330304  8 :6.377423
	  3 :4.151664  7 :7.477356
	  4 :9.003724  6 :7.477356
	  5 :6.377423    
\end{verbatim}
\item testg3: ./graph -{}-grow 3 -{}-gcol 3 -{}-gfrom 1 -{}-gto 10 -{}-goutput gen.txt

Program generuje graf o 3 wierszach, 3 kolumnach, z wagami w zakresie [1,10], spójność losowa, do pliku gen.txt.

Do pliku gen.txt zostały zapisane dane:
\begin{verbatim}
3 3
	  1 :9.744961  3 :4.068821
	  0 :9.744961  2 :2.057102  4 :7.917296
	  1 :2.057102  5 :4.975027
	  0 :4.068821  4 :6.742813  6 :3.497021
	  1 :7.917296  3 :6.742813  5 :8.205188  7 :8.080706
	  2 :4.975027  4 :8.205188  8 :2.629625
	  3 :3.497021  7 :8.844706
	  4 :8.080706  6 :8.844706  8 :7.227123
	  5 :2.629625  7 :7.227123
\end{verbatim}
\item testg4: ./graph -{}-grow -3 -{}-gcol 3 -{}-gfrom 1 -{}-gto 10 -{}-gconnect 1 -{}-goutput gen.txt

Test jest stworzony by sprawdzić, jak zachowa się program, gdy przekażemy ujemną liczbę wierszy lub kolumn. Musi ustawić wartość domyslną.

Program przekazał na stdout komunikat:
\begin{verbatim}
Liczba wierszy musi być dodatnia. Ustawiam wartość 10.
\end{verbatim}

Do pliku gen.txt dodane dane:
\begin{verbatim}
10 3
	  1 :9.521784  3 :1.701164
	  0 :9.521784  2 :5.748943  4 :8.938465
	  1 :5.748943  5 :2.530003
	  0 :1.701164  4 :7.268885  6 :8.841238
	  1 :8.938465  3 :7.268885  5 :4.114816  7 :5.534049
	  2 :2.530003  4 :4.114816  8 :3.170969
	  3 :8.841238  7 :7.033253  9 :2.314859
	  4 :5.534049  6 :7.033253  8 :3.393655  10 :7.541990
	  5 :3.170969  7 :3.393655  11 :5.012396
	  6 :2.314859  10 :7.201710  12 :7.579402
	  7 :7.541990  9 :7.201710  11 :3.206962  13 :2.010293
	  8 :5.012396  10 :3.206962  14 :3.690284
	  9 :7.579402  13 :1.820537  15 :8.809376
	  10 :2.010293  12 :1.820537  14 :6.245075  16 :3.823260
	  11 :3.690284  13 :6.245075  17 :2.400434
	  12 :8.809376  16 :2.119280  18 :2.688436
	  13 :3.823260  15 :2.119280  17 :5.210797  19 :8.767985
	  14 :2.400434  16 :5.210797  20 :6.244757
	  15 :2.688436  19 :1.701104  21 :8.289769
	  16 :8.767985  18 :1.701104  20 :6.945921  22 :6.450046
	  17 :6.244757  19 :6.945921  23 :7.228234
	  18 :8.289769  22 :8.475924  24 :3.718931
	  19 :6.450046  21 :8.475924  23 :6.069472  25 :2.590740
	  20 :7.228234  22 :6.069472  26 :8.252980
	  21 :3.718931  25 :8.240441  27 :8.623992
	  22 :2.590740  24 :8.240441  26 :9.567839  28 :1.634096
	  23 :8.252980  25 :9.567839  29 :6.165982
	  24 :8.623992  28 :4.580236
	  25 :1.634096  27 :4.580236  29 :7.835806
	  26 :6.165982  28 :7.835806
\end{verbatim}

\item test2: ./graph -{}-file gen.txt -{}-from 0 -{}-output sciezka.txt

Program wczytuje graf(może być wcześniej wygenerowany) z pliku gen.txt i szuka ścieżki od 0 węzła do maksymalnego. Ścieżka zapisana będzie do pliku sciezka.txt.

Przyjmijmy, że plik gen.txt zawiera:
\begin{verbatim}
3 3
	  1 :9.744961  3 :4.068821
	  0 :9.744961  2 :2.057102  4 :7.917296
	  1 :2.057102  5 :4.975027
	  0 :4.068821  4 :6.742813  6 :3.497021
	  1 :7.917296  3 :6.742813  5 :8.205188  7 :8.080706
	  2 :4.975027  4 :8.205188  8 :2.629625
	  3 :3.497021  7 :8.844706
	  4 :8.080706  6 :8.844706  8 :7.227123
	  5 :2.629625  7 :7.227123
\end{verbatim}

Do pliku sciezka.txt zostały dodane dane:
\begin{verbatim}
Najkrótsza ścieżka:
0 -9.744961- 1 -2.057102- 2 -4.975027- 5 -2.629625- 8
Długość ścieżki równa 19.406715
\end{verbatim}

Przyjmijmy, że plik gen.txt zawiera:
\begin{verbatim}
3 3
	  3 :8.550519
	  4 :6.943854

	  0 :8.550519  4 :7.743385
	  1 :6.943854  3 :7.743385  7 :1.404512

	  7 :8.996154
	  4 :1.404512  6 :8.996154

\end{verbatim}

Program przekazał na stderr:
\begin{verbatim}
Podany graf nie jest spójny. Przerywam działanie.
make: *** [Makefile:20: test2] Error 1
\end{verbatim}
\item test3: ./graph -{}-file gen.txt

Program przyjmuje tylko plik z grafem. Musi znaleźć ścieżkę od wierzchołka 0 do maksymalnego i wypisać na stdout.

Załózmy, że plik gen.txt zawiera:
\begin{verbatim}
3 3
	  1 :9.744961  3 :4.068821
	  0 :9.744961  2 :2.057102  4 :7.917296
	  1 :2.057102  5 :4.975027
	  0 :4.068821  4 :6.742813  6 :3.497021
	  1 :7.917296  3 :6.742813  5 :8.205188  7 :8.080706
	  2 :4.975027  4 :8.205188  8 :2.629625
	  3 :3.497021  7 :8.844706
	  4 :8.080706  6 :8.844706  8 :7.227123
	  5 :2.629625  7 :7.227123
\end{verbatim}
Na stdout wypisano:
\begin{verbatim}
Najkrótsza ścieżka: 
0 -9.744961- 1 -2.057102- 2 -4.975027- 5 -2.629625- 8
Długość ścieżki równa 19.406715
\end{verbatim}
\item test4: ./graph --from 0 --to 6 --output sciezka.txt

Test jest stworzony by sprawdzić, jak zachowa się program, gdy nie zostanie podany plik z danymi. Musi wyświetlić odpowiedni komunikat, wygenerować graf spójny o rozmiarach 10x10, o wagach w zakresie [0.1, 10], znaleźć ścieżke od wierzchołka 0 do 6 i zapisać do pliku sciezka.txt

Na stdout zostały przekazane następne dane:
\begin{verbatim}
./graph: nie został podany plik z danymi. Tworzę spójny graf o rozmiarach 10, 10, 
z wartościami wag w zakresie [0.010000, 10.000000].
10 10
	  1 :3.655630  10 :2.612224
	  0 :3.655630  2 :1.070027  11 :4.865590
	  1 :1.070027  3 :8.806976  12 :9.669014
	  2 :8.806976  4 :3.790733  13 :6.062454
	  3 :3.790733  5 :2.862130  14 :2.547590
	  4 :2.862130  6 :5.637390  15 :5.026725
	  5 :5.637390  7 :3.254020  16 :4.216804
	  6 :3.254020  8 :5.339291  17 :2.490274
	  7 :5.339291  9 :4.132861  18 :1.683471
	  8 :4.132861  19 :1.173396
	  0 :2.612224  11 :7.098354  20 :9.458073
	  1 :4.865590  10 :7.098354  12 :3.070200  21 :7.371739
	  2 :9.669014  11 :3.070200  13 :9.645168  22 :8.360334
	  3 :6.062454  12 :9.645168  14 :9.562849  23 :4.488464
	  4 :2.547590  13 :9.562849  15 :5.001660  24 :0.601119
	  5 :5.026725  14 :5.001660  16 :2.623777  25 :8.867515
	  6 :4.216804  15 :2.623777  17 :4.246750  26 :5.226001
	  7 :2.490274  16 :4.246750  18 :9.927542  27 :9.102340
	  8 :1.683471  17 :9.927542  19 :4.032977  28 :9.596557
	  9 :1.173396  18 :4.032977  29 :2.893073
	  10 :9.458073  21 :0.095431  30 :2.458686
	  11 :7.371739  20 :0.095431  22 :5.430663  31 :5.722821
	  12 :8.360334  21 :5.430663  23 :7.475411  32 :8.674683
	  13 :4.488464  22 :7.475411  24 :9.929625  33 :2.814702
	  14 :0.601119  23 :9.929625  25 :1.164957  34 :4.062486
	  15 :8.867515  24 :1.164957  26 :4.488173  35 :2.328353
	  16 :5.226001  25 :4.488173  27 :1.160841  36 :3.946247
	  17 :9.102340  26 :1.160841  28 :5.388552  37 :8.522580
	  18 :9.596557  27 :5.388552  29 :3.591415  38 :3.748887
	  19 :2.893073  28 :3.591415  39 :8.085429
	  20 :2.458686  31 :8.069879  40 :8.740547
	  21 :5.722821  30 :8.069879  32 :8.676548  41 :0.693655
	  22 :8.674683  31 :8.676548  33 :7.608062  42 :2.923298
	  23 :2.814702  32 :7.608062  34 :5.909656  43 :7.535604
	  24 :4.062486  33 :5.909656  35 :2.025637  44 :9.932633
	  25 :2.328353  34 :2.025637  36 :7.132161  45 :4.908711
	  26 :3.946247  35 :7.132161  37 :0.028063  46 :9.580847
	  27 :8.522580  36 :0.028063  38 :0.339374  47 :5.740884
	  28 :3.748887  37 :0.339374  39 :7.056258  48 :9.004057
	  29 :8.085429  38 :7.056258  49 :5.670509
	  30 :8.740547  41 :9.860960  50 :0.169014
	  31 :0.693655  40 :9.860960  42 :9.722995  51 :4.349134
	  32 :2.923298  41 :9.722995  43 :2.487366  52 :0.883836
	  33 :7.535604  42 :2.487366  44 :8.285380  53 :7.865918
	  34 :9.932633  43 :8.285380  45 :9.396416  54 :1.876795
	  35 :4.908711  44 :9.396416  46 :1.614805  55 :7.481845
	  36 :9.580847  45 :1.614805  47 :9.936674  56 :0.355352
	  37 :5.740884  46 :9.936674  48 :6.158393  57 :0.630329
	  38 :9.004057  47 :6.158393  49 :7.953414  58 :9.071690
	  39 :5.670509  48 :7.953414  59 :6.529985
	  40 :0.169014  51 :5.489018  60 :1.097328
	  41 :4.349134  50 :5.489018  52 :6.462618  61 :2.621179
	  42 :0.883836  51 :6.462618  53 :5.996038  62 :6.480681
	  43 :7.865918  52 :5.996038  54 :2.202025  63 :6.325412
	  44 :1.876795  53 :2.202025  55 :2.221565  64 :9.248283
	  45 :7.481845  54 :2.221565  56 :5.329469  65 :7.882074
	  46 :0.355352  55 :5.329469  57 :9.109244  66 :5.488482
	  47 :0.630329  56 :9.109244  58 :7.605069  67 :3.458377
	  48 :9.071690  57 :7.605069  59 :7.965848  68 :8.478905
	  49 :6.529985  58 :7.965848  69 :1.743758
	  50 :1.097328  61 :5.831767  70 :7.875321
	  51 :2.621179  60 :5.831767  62 :3.610553  71 :7.436572
	  52 :6.480681  61 :3.610553  63 :5.357166  72 :3.547226
	  53 :6.325412  62 :5.357166  64 :7.781923  73 :1.515558
	  54 :9.248283  63 :7.781923  65 :4.167555  74 :5.735337
	  55 :7.882074  64 :4.167555  66 :0.587249  75 :0.697540
	  56 :5.488482  65 :0.587249  67 :1.224355  76 :1.674577
	  57 :3.458377  66 :1.224355  68 :7.150158  77 :3.835533
	  58 :8.478905  67 :7.150158  69 :7.660615  78 :3.630839
	  59 :1.743758  68 :7.660615  79 :6.027559
	  60 :7.875321  71 :3.986027  80 :5.842404
	  61 :7.436572  70 :3.986027  72 :5.275842  81 :9.305496
	  62 :3.547226  71 :5.275842  73 :3.724478  82 :4.385086
	  63 :1.515558  72 :3.724478  74 :4.793978  83 :1.329548
	  64 :5.735337  73 :4.793978  75 :7.833463  84 :2.759826
	  65 :0.697540  74 :7.833463  76 :9.798453  85 :9.567221
	  66 :1.674577  75 :9.798453  77 :8.581593  86 :7.673774
	  67 :3.835533  76 :8.581593  78 :3.177773  87 :6.018165
	  68 :3.630839  77 :3.177773  79 :3.030940  88 :6.714999
	  69 :6.027559  78 :3.030940  89 :3.800088
	  70 :5.842404  81 :4.536499  90 :0.882554
	  71 :9.305496  80 :4.536499  82 :9.525425  91 :5.113748
	  72 :4.385086  81 :9.525425  83 :1.570094  92 :0.749780
	  73 :1.329548  82 :1.570094  84 :6.778324  93 :8.710252
	  74 :2.759826  83 :6.778324  85 :4.575313  94 :4.438939
	  75 :9.567221  84 :4.575313  86 :2.341090  95 :0.602872
	  76 :7.673774  85 :2.341090  87 :8.414966  96 :8.173494
	  77 :6.018165  86 :8.414966  88 :5.868714  97 :7.720462
	  78 :6.714999  87 :5.868714  89 :1.897972  98 :0.253799
	  79 :3.800088  88 :1.897972  99 :2.514440
	  80 :0.882554  91 :3.217520
	  81 :5.113748  90 :3.217520  92 :8.077262
	  82 :0.749780  91 :8.077262  93 :5.264266
	  83 :8.710252  92 :5.264266  94 :3.015973
	  84 :4.438939  93 :3.015973  95 :7.644483
	  85 :0.602872  94 :7.644483  96 :3.845859
	  86 :8.173494  95 :3.845859  97 :0.689748
	  87 :7.720462  96 :0.689748  98 :0.822256
	  88 :0.253799  97 :0.822256  99 :9.854024
	  89 :2.514440  98 :9.854024
\end{verbatim}
Do pliku sciezka.txt dodane dane:
\begin{verbatim}
Najkrótsza ścieżka: 
0 -3.655630- 1 -1.070027- 2 -8.806976- 3 -3.790733- 4 -2.862130- 5 -5.637390- 6
Długość ścieżki równa 25.822886
\end{verbatim}
\item test5: ./graph -{}-from 0 -{}-to 15 -{}-output sciezka.txt -{}-grow 2 -{}-gcol 2


Test jest przeprowadzany, żeby zobaczyć, jak zachowa się program, gdy numer wierzchołka końcowego będzie wykraczał poza zakres numerów wierzchołków grafu. Musi przyjąć najwięksy możliwy numer wierzchołka.

Na stdout zostały przekazane:
\begin{verbatim}
./graph: nie został podany plik z danymi. Tworzę spójny graf o rozmiarach 2, 2, 
z wartościami wag w zakresie [0.010000, 10.000000].
2 2
	  1 :9.016836  2 :1.360089
	  0 :9.016836  3 :9.087418
	  0 :1.360089  3 :4.407419
	  1 :9.087418  2 :4.407419
Numer węzła początkowego znajduje się za przedziałem grafu. Ustawiam wartość 3.
\end{verbatim}

Do pliku sciezka.txt:
\begin{verbatim}
Najkrótsza ścieżka: 
0 -1.360089- 2 -4.407419- 3
Długość ścieżki równa 5.767508
\end{verbatim}
\item test6: ./graph -{}-from -5 -{}-to 5 -{}-output sciezka.txt --grow 3 --gcol 3

Test służy do sprawdzania działania programu w przypadku, gdy zostanie podany ujemny numer wierzchołku. Program musi ustawić wartość tego wierzchołka jako 0.

Na stdout otrzymujemy:
\begin{verbatim}
./graph: nie został podany plik z danymi. Tworzę spójny graf o rozmiarach 3, 3,
z wartościami wag w zakresie [0.010000, 10.000000].
3 3
	  1 :0.017384  3 :8.733991
	  0 :0.017384  2 :9.364576  4 :8.515349
	  1 :9.364576  5 :8.395516
	  0 :8.733991  4 :5.120704  6 :6.379204
	  1 :8.515349  3 :5.120704  5 :3.054419  7 :4.072059
	  2 :8.395516  4 :3.054419  8 :3.311567
	  3 :6.379204  7 :6.759633
	  4 :4.072059  6 :6.759633  8 :1.798294
	  5 :3.311567  7 :1.798294
Numer węzła początkowego musi być większy lub równy 0. Ustawiam wartość 0.
\end{verbatim}
Do pliku sciezka.txt zostały dodane dane:
\begin{verbatim}
Najkrótsza ścieżka: 
0 -0.017384- 1 -8.515349- 4 -3.054419- 5
Długość ścieżki równa 11.587152
\end{verbatim}
\item test7: ./graph -{}-from 0 -{}-to 3 -{}-output sciezka.txt -{}-grow 2 -{}-gcol 3 -{}-gfrom 4 -{}-gto 12 -{}-gconnect 1 -{}-goutput gen.txt

Test przeznaczony do sprawdzenia poprawności działania algorytmu odnajdywania ścieżki i generatora grafu.

\item test8: ./graph

Test przeznaczony, by sprawdzić, jak zachowa się program przy niepodaniu żadnych argumentów. Musi wygenerować graf o rozmiarach 10x10, wagach w zakresie [0.1, 10], z losową spójnością, i przekazać na stdout.

Wynik na stdout:
\begin{verbatim}
10 10

	  11 :2.686477
	  3 :6.962296
	  2 :6.962296  4 :6.428726  13 :3.339158
	  3 :6.428726  14 :8.122333
	  6 :2.439876
	  5 :2.439876  7 :4.148008
	  6 :4.148008  17 :5.581391

	  19 :0.028774
	  11 :8.016655
	  1 :2.686477  10 :8.016655
	  22 :3.553451
	  3 :3.339158  14 :5.349312  23 :9.782359
	  4 :8.122333  13 :5.349312


	  7 :5.581391
	  28 :6.719558
	  9 :0.028774  29 :4.032725

	  22 :9.109316
	  12 :3.553451  21 :9.109316  23 :8.289380  32 :7.486189
	  13 :9.782359  22 :8.289380  24 :9.075430
	  23 :9.075430  25 :2.628881
	  24 :2.628881  35 :0.373052

	  37 :4.597610
	  18 :6.719558
	  19 :4.032725  39 :8.789516


	  22 :7.486189  33 :4.739009  42 :7.541546
	  32 :4.739009
	  35 :3.139264
	  25 :0.373052  34 :3.139264

	  27 :4.597610

	  29 :8.789516
	  41 :8.765130  50 :6.376369
	  40 :8.765130
	  32 :7.541546  43 :0.531902
	  42 :0.531902  44 :4.410621
	  43 :4.410621
	  46 :0.743160
	  45 :0.743160  56 :6.205323
	  48 :7.449568
	  47 :7.449568  49 :0.320291
	  48 :0.320291
	  40 :6.376369  60 :8.902086
	  52 :4.097223  61 :4.855210
	  51 :4.097223
	  54 :7.219384
	  53 :7.219384  55 :4.232222  64 :2.405176
	  54 :4.232222  56 :8.785330
	  46 :6.205323  55 :8.785330  57 :9.159727  66 :1.611258
	  56 :9.159727  58 :3.869703  67 :6.039311
	  57 :3.869703  68 :8.217539
	  69 :3.122824
	  50 :8.902086  70 :0.688998
	  51 :4.855210  62 :4.781214  71 :8.834890
	  61 :4.781214  63 :2.386874  72 :7.343029
	  62 :2.386874
	  54 :2.405176
	  66 :2.080839
	  56 :1.611258  65 :2.080839
	  57 :6.039311
	  58 :8.217539  69 :4.486197
	  59 :3.122824  68 :4.486197
	  60 :0.688998  71 :2.072516
	  61 :8.834890  70 :2.072516  72 :5.185341  81 :3.431125
	  62 :7.343029  71 :5.185341  73 :3.497840
	  72 :3.497840
	  84 :1.346539
	  85 :4.632048
	  77 :4.454826  86 :0.515034
	  76 :4.454826  78 :7.933454
	  77 :7.933454
	  89 :2.866110
	  81 :3.712940
	  71 :3.431125  80 :3.712940

	  84 :6.610847
	  74 :1.346539  83 :6.610847  85 :8.933577  94 :1.132275
	  75 :4.632048  84 :8.933577  86 :6.983580
	  76 :0.515034  85 :6.983580  87 :1.615628
	  86 :1.615628  97 :1.622864
	  89 :6.241969
	  79 :2.866110  88 :6.241969  99 :0.771039
	  91 :9.923871
	  90 :9.923871  92 :2.055055
	  91 :2.055055
	  94 :2.075123
	  84 :1.132275  93 :2.075123
	  96 :5.516315
	  95 :5.516315
	  87 :1.622864  98 :7.159846
	  97 :7.159846  99 :4.568931
	  89 :0.771039  98 :4.568931

\end{verbatim}
\end{itemize}

Testy, których nie da się zaimplementować do Makefile, to testy związane z poprawnością danych w pliku. Takie testy przeprowadziłyśmy ręcznie.

\begin{enumerate}
    \item Dane wejściowe: \texttt{./graph -{}-file dane.txt -{}-from 0 -{}-to 3 -{}-output sciezka.txt}
    
Plik \texttt{dane.txt} zawiera:
\begin{verbatim}
2 2
    1 :-5.74  2 :3.21
    3 :10.21
    3 :1.74
\end{verbatim}
Na wyjściu otrzymujemy komunikat \texttt{„ Waga krawędzi między węzłem 0 i 1 jest ujemna, równa -5.740000. Ustawiam wartość 5.740000.”}, przekazany na \texttt{stdout} i plik \texttt{sciezka.txt}, zawierający:
\begin{verbatim}
Najkrótsza ścieżka: 
0 -3.210000- 2 -1.740000- 3
Długość ścieżki równa 4.950000
\end{verbatim}
    \item Dane wejściowe: \texttt{./graph -{}-file dane.txt -{}-output sciezka.txt}

Plik \texttt{dane.txt} zawiera:
\begin{verbatim}
Nie graf!
\end{verbatim}
Na wyjściu otrzymujemy komunikat \texttt{”Błędny format pliku z grafem. Przerywam działanie.”}, przekazany na \texttt{stderr}.

    \item Dane wejściowe: \texttt{./graph -{}-file blednygraf.txt}
    
Plik \texttt{blednygraf.txt} zawiera:
\begin{verbatim}
2 3
        1 :1.0  4 :8.0
        0 :1.0  2 :3.0  4 :6.0
        1 :3.0  5 :7.0

        1 :6.0  5 :2.0  0 :8.0
        2 :7.0  4 :2.0
\end{verbatim}
Powyższy plik jest opisem poniższego grafu:
\begin{figure}
\centering
  \begin{tikzpicture}[node distance = 2cm, auto]
  \node [circle, draw, text centered] (zero) {\large \textbf{0}};
    \node [circle, draw, text centered, right of = zero] (one) {\large \textbf{1}};
      \node [circle, draw, text centered, right of = one] (two) {\large \textbf{2}};
        \node [circle, draw, text centered, below of = zero] (three) {\large \textbf{3}};
          \node [circle, draw, text centered, below of = one] (four) {\large \textbf{4}};
            \node [circle, draw, text centered, below of = two] (five) {\large \textbf{5}};
            
                \path [line] (zero) -- node{\small 1}(one);
                \path [line] (one) -- node{\small 3}(two);
                \path [line] (one) -- node{\small 6}(four);
                \path [line] (two) -- node{\small 7}(five);
                \path [line] (four) -- node{\small 2}(five);
                \path [line] (zero) -- node{\small 8}(four);
    \end{tikzpicture}
\end{figure}
\newline
Zgodnie z założeniem poprawności grafu, krawędzie mogą istnieć tylko między sąsiadującymi węzłami. Program wykrywa tę niepoprawność i wypisuje komunikat:
\begin{verbatim}
    Błędny format pliku z grafem. Między węzłem 0 a 4 
    nie może istnieć połączenie. Przerywam działanie.
\end{verbatim}

\end{enumerate}

\section{Błędy i wnioski}\label{header-n233}
Podczas pisania programu problemem okazało się niedoprecyzowanie wymagań dotyczących budowy grafu. Plik testowy \texttt{mygraph} o poniższej zawartości (tu widoczne pierwszych kilka linii pliku)
\begin{verbatim}
7 4
	 1 :0.8864916775696521  4 :0.2187532451857941 
	 5 :0.2637754478952221  2 :0.6445273453144537  0 :0.4630166785185348 
	 6 :0.8650384424149676  3 :0.42932761976709255  1 :0.6024952385895536 
	 7 :0.5702072705027322  2 :0.86456124269257 
	 8 :0.9452864187437506  0 :0.8961825862332892  5 :0.9299058855442358 
	 ...
	 \end{verbatim}
sugerował, że przyjmowany graf powinien być skierowany. Program \texttt{graph} obsługuje jednak tylko grafy nieskierowane, więc ten problem został zlikwidowany poprzez przekształcenie grafu skierowanego w nieskierowany dzięki duplikacji wag pomiędzy konkretnymi węzłami. Wczytując plik za pomocą funkcji \texttt{read\_graph}, a następnie wypisując go przy użyciu \texttt{write\_graph} otrzymujemy już odpowiedni graf: 
\bigskip
\begin{verbatim}
7 4
	1 :0.463017 4 :0.896183 
	0 :0.463017 2 :0.602495 5 :0.595644 
	1 :0.602495 3 :0.864561 6 :0.200570 
	2 :0.864561 7 :0.797053 
	0 :0.896183 5 :0.449257 8 :0.750068
	...
\end{verbatim}
Program można przekształcić tak, aby przyjmował graf skierowany, poprzez usunięcie linijki 47 w pliku \texttt{graph.c} (graph->weights[neighbor\_index * iter + i] = weight;). Wtedy graf nie będzie duplikował wagi.

Inną problematyczną kwestią była próba dostosowania funkcji \texttt{read\_graph} w ten sposób, aby funkcja interpretowała dane poprawnie pomimo niepełnej zgodności z wymogami formatu pliku. Początkowo funkcja read zawierała warunki określające ilość iteracji w linii,
\begin{verbatim}
if (n != 1 && m != 1){
if (i==0 || i==iter-1 || i==m-1 || i==iter-m) //warunek na 2 sąsiadów
        k = 2;
else if ((i%m == 0 || i%m == m-1 || (i>0 && i<m) || (i>iter-m && i<iter))) 
    //warunek na 3 sąsiadów
    k = 3;
else //w innym przypadku będzie 4 sąsiadów
    k = 4;
		}

if (n==1 || m==1){
	if (i==0 || i==iter-1)
    k = 1;
	else
		k = 2;
		}
		\end{verbatim}
		jednak kod nie mógł zostać użyty ze względu na to, że wierzchołki w grafie spójnym niekoniecznie muszą być połączone ze wszystkimi swoimi sąsiadami. Z tego powodu w funkcji wczytującej należało zaimplementować pętle nieskończone (ponieważ nie jest wiadomo ile iteracji w linii należy wykonać; wyjście z pętli następuje dopiero po wykryciu znaku nowej linii), co zwiększyło ryzyko pojawienia się błędów przy zapisywaniu grafu do struktury. Gdyby była znana liczba iteracji, użytkownik nie musiałby przestrzegać wymogów odpowiedniej ilości spacji między konkretnymi parametrami w pliku.
\end{document}
