\documentclass[]{article}
\usepackage{lmodern}
\usepackage{amssymb,amsmath}
\usepackage{ifxetex,ifluatex}
\usepackage{fixltx2e} % provides \textsubscript
\ifnum 0\ifxetex 1\fi\ifluatex 1\fi=0 % if pdftex
  \usepackage[T1]{fontenc}
  \usepackage[utf8]{inputenc}
\else % if luatex or xelatex
  \ifxetex
    \usepackage{mathspec}
  \else
    \usepackage{fontspec}
  \fi
  \defaultfontfeatures{Ligatures=TeX,Scale=MatchLowercase}
\fi
% use upquote if available, for straight quotes in verbatim environments
\IfFileExists{upquote.sty}{\usepackage{upquote}}{}
% use microtype if available
\IfFileExists{microtype.sty}{%
\usepackage[]{microtype}
\UseMicrotypeSet[protrusion]{basicmath} % disable protrusion for tt fonts
}{}
\PassOptionsToPackage{hyphens}{url} % url is loaded by hyperref
\usepackage[unicode=true]{hyperref}
\hypersetup{
            pdftitle={Generacja grafu i odnalezienie najkrótszej ścieżki pomiędzy węzłami: graph},
            pdfauthor={Ulyana Petrashevich, Inga Maziarz}
            pdfborder={0 0 0},
            breaklinks=true}
\urlstyle{same}  % don't use monospace font for urls
\IfFileExists{parskip.sty}{%
\usepackage{parskip}
}{% else
\setlength{\parindent}{0pt}
\setlength{\parskip}{6pt plus 2pt minus 1pt}
}
\setlength{\emergencystretch}{3em}  % prevent overfull lines
\providecommand{\tightlist}{%
  \setlength{\itemsep}{0pt}\setlength{\parskip}{0pt}}
\setcounter{secnumdepth}{0}
% Redefines (sub)paragraphs to behave more like sections
\ifx\paragraph\undefined\else
\let\oldparagraph\paragraph
\renewcommand{\paragraph}[1]{\oldparagraph{#1}\mbox{}}
\fi
\ifx\subparagraph\undefined\else
\let\oldsubparagraph\subparagraph
\renewcommand{\subparagraph}[1]{\oldsubparagraph{#1}\mbox{}}
\fi

% set default figure placement to htbp
\makeatletter
\def\fps@figure{htbp}
\makeatother


\title{\texttt{Specyfikacja funkcjonalna}\\Generacja grafu i odnalezienie najkrótszej ścieżki pomiędzy węzłami - \texttt{graph}}
\author{Ulyana Petrashevich, Inga Maziarz}
\date{23.02.2022}

\begin{document}
\maketitle

\section{Cel projektu}\label{header-n231}

Program \texttt{graph} ma na celu znalezienie najkrótszej ścieżki między dwoma wybranymi punktami. Program może wczytywać graf z pliku o określonym formacie lub wygenerować graf o zadanej liczbie kolumn i wierszy z losowymi wartościami wag krawędzi w określonym przedziale. Sprawdzana jest spójność grafu. Program działa w trybie wsadowym.

\section{Dane wejściowe}\label{header-n233}

Program przyjmuje dane o grafie z pliku tekstowego, w którym są opisane parametry węzłów:

\begin{itemize}
\item
  liczba wierszy i kolumn węzłów: liczby całkowite \textgreater{} 0;
\item
  numer węzła, do którego prowadzi krawędź od węzła obecnego: numer obecnego węzła odpowiada numerowi linii w pliku - 1, nie licząc linii opisującej liczbę kolumn i wierszy. Przyjęta jest konwencja, że węzły przyjmują indeksy od 0 do n - 1 (gdzie n to sumaryczna liczba węzłów) i są numerowane kolejno od lewej do prawej; 
\item
  waga krawędzi: liczba rzeczywista \textgreater{} 0.
\end{itemize}

W pierwszej linii pliku podana jest ilość wierszy i kolumn grafu. Parametry do jednego węzła są definiowane poniżej w jednej linii. Numer węzła docelowego jest oddzielany od wagi spacją i znakiem \texttt{:}. Parametry dla jednego węzła są oddzielane dwiema spacjami.

W przypadku generowania grafu przez program, parametry są przekazywane za pomocą argumentów wywołania programu. Jeżeli parametry nie zostaną określone, wygenerowany zostanie graf bazujący na domyślnych wartościach (10 wierszy, 10 kolumn, przedział wagowy krawędzi [0.01, 10.0]).



Przykład danych wejściowych:

\begin{verbatim}
2 3
	 1 :0.8864916775696521  2 :0.2187532451857941 
	 0 :0.2637754478952221  5 :0.5486221194511974  2 :0.25413610146892474
	 3 :0.5380334165340379
	 2 :0.5486221194511974  5 :0.25413610146892474
	 3 :0.31509645530519625  0 :0.40326574227480094  5 :0.8901867253885717
	 4 :0.44272335750313074
\end{verbatim}
Wyżej jest zdefiniowany graf o rozmiarze 2 wierszy i 3 kolumn. Dla węzła numer 0 istnieją przejścia do węzła 1 o wadze 0.8865 i do węzła 2 o wadze 0.2188.

\section{Dane wyjściowe}\label{header-n279}
Aby wyznaczenie najkrótszej ścieżki zakończyło się pomyślnie, graf musi być spójny. Po sprawdzeniu poprawności grafu, program wypisuje najkrótszą znalezioną ścieżkę w postaci\texttt{ numer węzła początkowego – waga – numer węzła pośredniego-...-numer węzła końcowego}. Wypisywana jest sumaryczna długość znalezionej ścieżki.

W przypadku generowania grafu, program zapisuje parametry węzłów do pliku tekstowego w postaci analogicznej do danych wejściowych: liczby wierszy i kolumn, numery węzłów, do których prowadzi krawędź od węzła odpowiadającego numerowi linii – 1, nie licząc linii z liczbą kolumn i wierszy.

Przykłady danych wyjściowych:

\begin{verbatim}
Najkrótsza ścieżka:
0 -0.1635472537235685- 3 -0.6524342534143475- 5
Długość ścieżki równa 0.815981507137916
\end{verbatim}
Program znalazł najkrótszą ścieżkę od węzła 0 do węzła 5 i wyznaczył długość tej ścieżki.

\begin{verbatim}
3 1
    1 :2.4562823562334748  2: 7.2677219245832457 
    2: 5.3847356384723346
    0: 0.4537284536532248
\end{verbatim}
Program utworzył graf o 3 wierszach i 1 kolumnie węzłów. Wagi wygenerowane w zakresie [0.01, 10.0]
\section{Argumenty wywołania programu}\label{header-n256}

Program \texttt{graph} akceptuje następujące argumenty wywołania:

\begin{itemize}
\item
  \texttt{-file\ filename} nazwa pliku z danymi;
\item
  \texttt{-from\ k} podawany jest numer węzła początkowego w ścieżce; domyślnie \texttt{k\ =\ 0};
\item
  \texttt{-to\ l} podawany jest numer węzła końcowego w ścieżce; domyślnie \texttt{l\ =\ maksymalny numer w grafie};
\item
  \texttt{-output\ filename} nazwa pliku, w który będzie zapisywana ścieżka; domyślnie \texttt{stdout};
\item
  \texttt{-grow\ n} określa liczbę wierszy generowanego grafu; domyślnie \texttt{n\ =\ 10};
\item
  \texttt{-gcol\ m} określa liczbę kolumn w generowanym grafie; domyślnie \texttt{m\ =\ 10};
\item
  \texttt{-gfrom\ x} przyjmuje granicę dolną zakresu wartości losowanych wag, domyślnie \texttt{x\ =\ 0.01};
\item
  \texttt{-gto\ y} przyjmuje granicę górną zakresu wartości losowanych wag, domyślnie \texttt{y\ =\ 10.0};
\item
  \texttt{-gconnect\ s} przyjmuje flagę na spójność generowanego grafa. 0 - graf niespójny, 1 - graf spójny, 2 - graf losowy;
\item
  \texttt{-goutput\ filename} nazwa pliku, w którym będą zapisane parametry wygenerowanego grafu, domyślnie \texttt{stdout}.
\end{itemize}

Przykładowe wywołania programu:

\begin{itemize}
\item
  \texttt{./graph\ -grow\ 3\ -gcol\ 2\ -gfrom\ 4.5\ -gto\ 10\ -goutput\ graph.txt}
  , program utworzy graf o 3 wierszach, 2 kolumnach i wagach w przedziale \texttt{[} 4.5; 10.0\texttt{]};
\item
  \texttt{./graph\ -file\ graph.txt\ -from\ 2\ -to\ 15\ -output\ sciezka}
  , program wczyta dane z pliku \texttt{graph.txt}, znajdzie najkrótszą ścieżkę pomiędzy węzłami 2 i 15 i zapisze wynik do pliku \texttt{sciezka}. 
\end{itemize}

\section{Teoria}\label{header-n279}

Graf jest \textbf{spójny}, jeżeli dla każdej pary węzłów istnieje łącząca je ścieżka. Program sprawdza spójność grafu za pomocą algorytmu przeszukiwania grafu wszerz (BFS). Działanie algorytmu ma następne kroki:
\begin{itemize}
\item
  Zaczynamy od węzła początkowego. Zaznaczamy go jako odwiedzony i dodajemy do kolejki wszystkie węzły, z którymi jest powiązany, w kolejności od węzła z najmniejszym indeksem.
\item
  Odwiedzamy następny węzeł w kolejce. Dodajemy do kolejki wszystkie węzły z nim powiązane i jeszcze nieodwiedzone.
\item
  Powtarzamy poprzedni krok, aż kolejka będzie pusta. Jeżeli wszystkie węzły w grafie zostały odwiedzone, to można stwierdzić, że graf jest spójny.
\end{itemize}

Poszukiwanie najkrótszej ścieżki między zadanymi punktami program będzie realizował przez \textbf{algorytm Dijkstry}. Algorytm polega na odnajdowaniu najkrótszych ścieżek od zadanego węzła do każdego innego. 
\begin{itemize}
\item
  Dla każdego węzła ustawiamy długość ścieżki na nieskończoność lub wartość, która do niej dąży; długości przy węźle początkowym nadajemy wartość 0. 
\item
  Oznaczamy węzeł jako odwiedzony. Dla każdego węzła połączonego z początkowym, przypisujemy długość równą wadze krawędzi ich łączących.
\item
  Z nieodwiedzonych węzłów znajdujemy węzeł o najmniejszej przepisanej długości. Oznaczamy go jako odwiedzony. Dla każdego węzła sąsiadującego z obecnym liczymy wartość „długość przy obecnym węźle + waga krawędzi łączącej”. Jeżeli znaleziona wartość jest mniejsza niż przypisana do sąsiadującego węzła, podmieniamy ją.
\item
  Powtarzamy poprzedni krok, aż zostaną odwiedzone wszystkie węzły. Po zakończeniu każdy węzeł będzie miał przypisaną długość najkrótszej ścieżki od węzła początkowego. Samą ścieżkę możemy odtworzyć od końca, jeżeli przy każdym przypisaniu węzłowi nowej długości będziemy zapamiętywali numer poprzedniego węzła.
\end{itemize}

\section{Komunikaty błędów}\label{header-n281}

W przypadku, gdy znaczące dane nie są podane, program \texttt{graph} stara się nadawać im przypisane domyślne znaczenia. Lecz w przypadkach, gdy dane nie można zastąpić lub są podane błędnie, będą wyświetlane odpowiednie komunikaty.

\begin{enumerate}
\def\labelenumi{\arabic{enumi}.}
\item
  Podany plik nie można otworzyć:
  \texttt{Nie\ udało\ się\ otworzyć\ plik\ \emph{filename}.\ Przerywam\ działanie.}
\item
  Nie podano pliku z danymi:
  \texttt{Nie\ został\ podany\ plik\ z\ danymi.\ Tworzę\ graf\ o\ rozmiarach\ \emph{n},\ \emph{m},\ z\ wartościami\ wag\ w\ zakresie\ [\emph{x},\ \emph{y}].}
\item
  Nie podano liczby wierszy/kolumn:
  \texttt{Nie\ podano\ liczby\ wierszy\texttt{/}kolumn.\ Przerywam\ działanie. }
\item
  Podany numer początkowego węzła mniejszy 0:
  \texttt{Numer\ węzła\ początkowego\ musi\ być\ większy\ lub\ równy\ 0.\ Ustawiam\ wartość\ 0.}
\item
  Podany numer początkowego węzła leży za przedziałem grafu:
  \texttt{Numer\ węzła\ początkowego\ znajduje\ się\ za\ przedziałem\ grafu.\ Ustawiam\ wartość\ \emph{n * m - 1}.}
\item
  Podany numer końcowego węzła mniejszy 0:
  \texttt{Numer\ węzła\ końcowego\ musi\ być\ większy\ lub\ równy\ 0.\ Ustawiam\ wartość\ 0.}
\item
  Podany numer końcowego węzła leży za przedziałem grafu:
  \texttt{Numer\ węzła\ końcowego\ znajduje\ się\ za\ przedziałem\ grafu.\ Ustawiam\ wartość\ \emph{n * m - 1}.}
\item
  Podany graf nie jest spójny:
  \texttt{Podany\ graf\ nie\ jest\ spójny.\ Przerywam\ działanie.}
\item
  Podana liczba wierszy węzłów generowanego grafu mniejsza/równa 0:
  \texttt{Liczba\ wierszy\ musi\ być\ większa\ 0.\ Ustawiam\ wartość\ 10.}
\item
  Podana liczba kolumn węzłów generowanego grafu mniejsza/równa 0:
  \texttt{Liczba\ kolumn\ musi\ być\ większa\ 0.\ Ustawiam\ wartość\ 10.}
\item
  Dolna granica zakresu losowanych wag mniejsza/równa 0:
  \texttt{Dolna\ granica\ przedziału\ losowanych\ wartości\ wag\ musi\ być\ większa\ 0.\ Ustawiam\ wartość\ 0.01.}
\item
  Górna granica zakresu losowanych wag mniejsza/równa 0:
  \texttt{Górna\ granica\ przedziału\ losowanych\ wartości\ wag\ musi\ być\ większa\ 0.\ Ustawiam\ wartość\ \emph{x + 1}.}
\end{enumerate}

\end{document}
